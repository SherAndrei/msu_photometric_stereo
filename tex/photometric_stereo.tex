\section{Photometric stereo}

\subsection{Постановка задачи}

Задача, которую решает Photometric stereo, довольно проста и интересна:
если у нас имеется некоторый набор фотографий зафиксированного объекта,
сделанных под различными источниками света, можем ли восстановить трехмерную
модель этого объекта?

\begin{figure}[h]
  \centering
  \begin{tikzpicture}
    \path (-5,5) pic[rotate=310,transform shape] (A) {lighter} node[right]{Источник 1};
    \path (2,6) pic[rotate=250,transform shape] (B) {lighter} node[right]{Источник 2};
    \path (4,3) pic[rotate=220,transform shape] (C) {lighter} node[right]{Источник 3};
    \path (0,7) pic[rotate=180,transform shape] {camera} node[above]{Камера};

    \filldraw[pattern=north east lines] let \n1 = {(1cm)+rand*(2)} in
    +(0:\n1)
    \foreach \a in {10,20,...,350} {
    let \n1 = {(1cm)+rand*(2)} in
    -- +(\a:\n1) coordinate (Shape\a)
    } -- cycle node[right]{3D объект};

    \draw[dashed,color=yellow] (A-head) -- (Shape130);
    \draw[dashed,color=yellow] (B-head) -- (Shape80);
    \draw[dashed,color=yellow] (C-head) -- (Shape40);

  \end{tikzpicture}
  \caption{Постановка задачи}
\end{figure}

Для простоты сделаем два предположения: рассматриваются поверхности, удовлетворяющие
закону Ламберта, и источники света находятся на довольно большом расстоянии.

\subsection{Восстановление нормалей}

Из предыдущих разделов мы определили, что яркость изображения для поверхностей,
удовлетворяющих закону Ламберта, можно выразить как
\[I=\frac{\beta}{\pi}\cos\theta=\frac{\beta}{\pi}\mathbf{s}^T\norm\]
где $\beta$ - коэффициент диффузии, закладывающий в себя альбедо. Предположим, что
имеется три различных точечных источника света $\mathrm{s}_1$, $\mathrm{s}_2$, $\mathrm{s}_3$.
Сделав три фотографии используя эти источники, мы можем записать следующее выражение
\[I_i=\frac{\beta}{\pi}\mathrm{s}^T_i\norm,\ i=1,2,3\]
Так как положение камеры и объекта остается неизменным, каждый пиксель на трех фотографиях
соответствует единственной точке в трехмерном пространстве. Это значит, что для конкретного
пикселя, $I_i$ -- функция, зависящая от направления источника света, так как $\beta$ и $\norm$
не меняются. Таким образом, мы можем переписать предыдущее выражение в матричной форме
следующим образом:
\[\mathrm{I}=\left[\begin{array}{c}
      I_1 \\ I_2 \\ I_3
    \end{array}\right] = \frac{\beta}{\pi}\left[\begin{array}{c}
      \mathrm{s}^T_1 \\ \mathrm{s}^T_2 \\ \mathrm{s}^T_3
    \end{array}\right]\norm = \frac{\beta}{\pi}\mathrm{S}\norm\]
где $S$ - матрица $3\times3$. Так как направления источников света нам известны, то
\[\frac{\beta}{\pi}\norm=\mathrm{S}^{-1}\mathrm{I}\]
при условии, что матрица $\mathrm{S}$ обратима.
\begin{remark}
  Обратим внимание, что использовать для алгоритма фотографии какого-либо объекта
  с таким источником освещения как солнце в разные временные промежутки -- не самая лучшая
  идея, так как любые три вектора направления такого источника света будут почти линейно
  зависимы, что сразу влечет необратимость матрицы $\mathrm{S}$.
\end{remark}
Решив эту систему уравнений мы одновременно сможем посчитать как и нормали поверхности
$\norm=\frac{\mathrm{S}^{-1}I}{\Vert\mathrm{S}^{-1}I\Vert}$, так и
коэффициент диффузии $\beta=\pi\Vert\mathrm{S}^{-1}I\Vert$.

Мы можем получить более точные результаты, используя $k > 3$ источника света.
\[\mathrm{I}=\left[\begin{array}{c}
      I_1 \\ I_2 \\ \vdots \\ I_k
    \end{array}\right] = \frac{\beta}{\pi}\left[\begin{array}{c}
      \mathrm{s}^T_1 \\ \mathrm{s}^T_2 \\ \vdots \\ \mathrm{s}^T_k
    \end{array}\right]\norm = \frac{\beta}{\pi}\mathrm{S}\norm\]
В таком случае $\mathrm{S}$ является матрицей $k\times3$, то есть не является квадратной,
а значит ее нельзя обратить. Решить эту проблему возможно воспользовавшись методом
наименьших квадратов:
\[\mathrm{S}^T\mathrm{I}=\frac{\beta}{\pi}\underset{3\times3}{\mathrm{S}^T\mathrm{S}}\norm\Rightarrow\frac{\beta}{\pi}\norm=(\mathrm{S}^T\mathrm{S})^{-1}\mathrm{S}^T\mathrm{I}\]

\subsection{Восстановление поверхности из нормалей}

Мы получили нормали поверхностей, теперь нам требуется восстановить саму поверхность.
Как восстановить поверхность из нормалей? Вспомним, что такое нормаль к поверхности.
Представим нашу поверхность как функцию зависящую от двух параметров: $z=f(x,y)$.
Обозначим $p=-\frac{\partial z}{\partial x}$, $q=-\frac{\partial z}{\partial y}$.
Тогда нормаль в точке $(x_0,y_0)$ будет иметь вид
\[\norm=\left(\frac{1}{\sqrt{1+p^2+q^2}}\left[p,q,1\right]\right)_{(x_0,y_0)}\]

Для получения трехмерного объекта требуется решить обратную задачу:
проинтегрировать наш набор нормалей.

Зафиксируем точку $(x_0, y_0)$ как начало отсчета, возьмем $z(x_0,y_0)=0$.
Тогда мы можем посчитать интеграл по любому направлению следующим образом:
\[z(x,y)=z(x_0,y_0)+\int_{(x_0,y_0)}^{(x,y)}-(pdx+qdy)\]

Самая простая идея получения итоговой поверхности - это интегрирование
сначала в направлении одной оси координат, а затем в направлении второй.

\begin{algorithmic}[1]
  \State $z(0,0) \gets 0$;
  \For{$y=0,H-1$}
  \State $z(0,y) \gets z(0,y-1)-q(0,y)$;
  \EndFor
  \For{$y=0,H-1$}
  \For{$x=1,W-1$}
  \State $z(x,y) \gets z(x-1,y)-p(x,y)$;
  \EndFor
  \EndFor
\end{algorithmic}

Такой наивный метод чаще всего приведет к шуму в полученных данных,
так как никто не обещает, что значения в конечных точках направлений
будут совпадать. В таких случаях можно брать среднее по каждому направлению
и надеяться, что количество шумов уменьшится.

Возникает идея минимизировать погрешность между посчитанным нами
градиентом $(p,q)$ и градиентом приближенной поверхности $z(x,y)$.
В качестве меры ошибки рассмотрим
\begin{equation}\label{error_measure}
  D=\underset{\Omega}{\int\int}\left(\frac{\partial z}{\partial x}+p\right)^2+\left(\frac{\partial z}{\partial y}+q\right)^2\d x\d y
\end{equation}
где $\Omega$ - фотография, $\frac{\partial z}{\partial x}$ и $\frac{\partial z}{\partial y}$ - градиенты
приближенной поверхности.

Для нахождения поверхности $z(x,y)$, которая минимизирует $D$, будем использовать
алгоритм Франкорт-Челлаппа.

Положим $Z(u,v)$, $P(u,v)$ и $Q(u,v)$ дискретными преобразованиями Фурье
$z(x,y)$, $p(x,y)$ и $q(x,y)$ соответственно. Тогда
\[ z(x,y) = \int\int_{-\infty}^{+\infty}Z(u,v)e^{i2\pi(ux+vy)}\d u\d v \]
\[ p(x,y) = \int\int_{-\infty}^{+\infty}P(u,v)e^{i2\pi(ux+vy)}\d u\d v \]
\[ q(x,y) = \int\int_{-\infty}^{+\infty}Q(u,v)e^{i2\pi(ux+vy)}\d u\d v \]
Подставим $z(x,y)$, $p(x,y)$ и $q(x,y)$ в \eqref{error_measure}.
Найдем $Z(u,v)$ с помощью $\frac{\partial D}{\partial Z}=0$.
В итоге получим
\[ \hat{Z}(u,v)=\frac{iuP(u,v)+ivQ(u,v)}{u^2+v^2} \]
Для восстановления искомой поверхности воспользуемся обратным преобразованием
Фурье и получим $\hat{z}(x,y)$.

\newpage

\subsection{Проблемы, возможные решения}

\begin{itemize}
  \item Не находясь в лабораторных условиях довольно затруднительно посчитать
        направление источников света. Для подсчета используют дополнительный объект
        для калибровки, например, шар из хрома. Так как он обладает хорошими
        отражательными способностями, то зная его размер, можно довольно просто
        определить направление источника света.
  \item Чаще всего мы не знаем об отражательных свойствах рассматриваемого
        объекта, а значит и не можем выписать для него значения ДФОС. Что делать в таком случае?
        В таких случаях используют дополнительный объект заранее известной формы
        и сделанный из того же материала или покрытый тем же материалом, что и рассматриваемый объект.
        Этот метод имеет название "Calibration based Photometric Stereo".
  \item Рассмотри в качестве объекта глиняный горшок, и расположим камеру ровно над ним.
        В подходе Photometric stereo предполагается, что точка на поверхности
        получает свет только от одного источника, тогда как при попадании свет
        в глиняном горшке будет распростроняться по всей его внутренней поверхности.
        Это приведет к тому, что значение альбедо для каждой точки будет
        больше, чем является на самом деле. Предлагается
        итеративно восстанавливать корректную форму, принимая во внимание
        взаимоотражения внутри горшка.
\end{itemize}
